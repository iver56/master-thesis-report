% !TEX encoding = UTF-8 Unicode
%!TEX root = thesis.tex
% !TEX spellcheck = en-US
%%=========================================
\addcontentsline{toc}{section}{Abstract}
\section*{Abstract}
Cross-adaptive audio effects have many applications within music technology, including for automatic mixing and live music. Commonly used methods of signal analysis capture the acoustical and mathematical features of the signal well, but struggle to capture the musical meaning. Together with the vast number of possible signal interactions, this makes manual exploration of signal mappings difficult and tedious. This project investigates Artificial Intelligence (AI) methods for finding useful signal interactions in cross-adaptive audio effects. A system for doing signal interaction experiments and evaluating their results has been implemented. Since the system produces lots of output data in various forms, a significant part of the project has been about developing an interactive visualization tool which makes it easier to evaluate results and understand what the system is doing. The overall goal of the system is to make one sound similar to another by applying audio effects. The parameters of the audio effects are controlled dynamically by the features of the other sound. The features are mapped to parameters by using evolved neural networks. NeuroEvolution of Augmenting Topologies (NEAT) is used for evolving neural networks that have the desired behavior. Several neuroevolution configurations have been evaluated.
