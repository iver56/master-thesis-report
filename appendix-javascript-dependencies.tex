% !TEX encoding = UTF-8 Unicode
%!TEX root = thesis.tex
% !TEX spellcheck = en-US
%%=========================================

\chapter{JavaScript libraries in interactive visualization application}
\label{appendix:javascript_dependencies}

\begin{center}
\begin{longtable}{p{3.3cm} p{13cm}}
\caption[JavaScript dependencies]{JavaScript libraries} \label{tab:javascript_dependencies} \\

\hline \multicolumn{1}{l}{\textbf{Name}} & \multicolumn{1}{l}{\textbf{Description}} \\ \hline 
\endfirsthead

\multicolumn{2}{c}%
{{\bfseries \tablename\ \thetable{} -- continued from previous page}} \\
\hline \multicolumn{1}{l}{\textbf{Name}} & \multicolumn{1}{l}{\textbf{Description}} \\ \hline 
\endhead

\hline \multicolumn{2}{r}{{Continued on next page}} \\ \hline
\endfoot

\hline \hline
\endlastfoot

NodeJS & Used for serving the application and pushing results to the application via websockets whenever new data becomes available \\
\hline
AngularJS & Application framework that makes it easy to build Single-Page Applications \\
\hline
Angular-material & User Interface (UI) Component framework. Makes it easy to add UI elements, such as buttons and sliders, that look nice and have great usability. \\
\hline
Color-brewer & Various sets of colors that are useful for visualizing data \\
\hline
Cubism & Time series visualization in the form of horizon charts, which reduce vertical space without losing resolution. This is useful when there are many variables to visualize simultaneously in a limited vertical space. \\
\hline
n3-line-chart & Used for line charts and histograms. Is nicely integrated with AngularJS and features some useful interactions. Depends on D3.js \\
\hline
NVD3 & Chart components for D3.js. Used for creating stacked area chart (species plot) \\
\hline
Debounce, limit & Used for throttling the refresh rate of computationally demanding actions \\
\hline
Sigma & Used for visualizing neural networks. Features zooming, panning and rotating. \\
\hline
Wavesurfer & Works as an audio player that also visualizes the waveform of the sound that is played. The user can click on the waveform to seek to that position. \\
\hline
jQuery & Makes it easier to do Document Object Model (DOM) manipulation \\

\end{longtable}
\end{center}