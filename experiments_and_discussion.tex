% !TEX encoding = UTF-8 Unicode
%!TEX root = thesis.tex
% !TEX spellcheck = en-US
%%=========================================
\chapter{Experiments and Discussion}
\label{chapter:experiments_and_discussion}

Five experiments have been conducted in an attempt to find good ways to use neuroevolution for finding useful mappings from audio features to audio effect parameters. In each experiment, different neuroevolution configurations are compared to find what works better. Due to the random nature of genetic algorithms, results vary from run to run. To deal with this, each configuration gets multiple runs (with different Pseudo Random Number Generator (PRNG) seeds), and results from the runs are aggregated and presented in various figures. Table \ref{tab:experiments_overview} shows a rough overview of the experiments.

\begin{center}
\begin{longtable}{p{2.3cm} p{13cm}}
\caption[Overview of experiments]{Overview of experiments} \label{tab:experiments_overview} \\

\hline \multicolumn{1}{l}{\textbf{Experiment}} & \multicolumn{1}{l}{\textbf{Description}} \\ \hline 
\endfirsthead

\multicolumn{2}{c}%
{{\bfseries \tablename\ \thetable{} -- continued from previous page}} \\
\hline \multicolumn{1}{l}{\textbf{Experiment}} & \multicolumn{1}{l}{\textbf{Description}} \\ \hline 
\endhead

\hline \multicolumn{2}{r}{{Continued on next page}} \\ \hline
\endfoot

\hline \hline
\endlastfoot

\midrule
  1 & Find a good combination of mutation rate and crossover rate \\
\midrule
  2 & Find a good value for structural mutation parameters \\
\midrule
  3 & Apply data augmentation to the target sound, and see if the result generalizes better \\
\midrule
  4 & Compare sets of audio features used in the fitness function. Analyze data from the neuroevolution process. \\
\midrule
  5 & Compare networks of audio effects with individual audio effects. \\
\end{longtable}
\end{center}

\section{General Configuration}
Table \ref{tab:general_configuration} shows the parameters used unless otherwise stated in individual experiments. The table is not exhaustive. A number of NEAT parameters were left at their respective default values\footnote{\url{https://github.com/peter-ch/MultiNEAT/blob/master/src/Parameters.cpp\#L42}}, set by \citeauthor{multineat}, the authors of MultiNEAT.

\begin{center}
\begin{longtable}{p{10cm} p{4cm}}
\caption[General experiment configuration]{General experiment configuration} \label{tab:general_configuration} \\

\hline \multicolumn{1}{l}{\textbf{Parameter}} & \multicolumn{1}{l}{\textbf{Value}} \\ \hline 
\endfirsthead

\multicolumn{2}{c}%
{{\bfseries \tablename\ \thetable{} -- continued from previous page}} \\
\hline \multicolumn{1}{l}{\textbf{Parameter}} & \multicolumn{1}{l}{\textbf{Value}} \\ \hline 
\endhead

\hline \multicolumn{2}{r}{{Continued on next page}} \\ \hline
\endfoot

\hline \hline
\endlastfoot

\midrule
  Population size & 20 \\
\midrule
  Add neuron probability & 0.01 \\
\midrule
  Remove neuron probability & 0.01 \\
\midrule
  Add link probability & 0.01 \\
\midrule
  Remove link probability & 0.01 \\
\midrule
  Elite fraction & 0.1 \\
\midrule
  Survival rate & 0.25 \\
\midrule
  Allow clones & Yes \\
\midrule
  Selection method & Tournament selection \\
\midrule
  Hidden activation function & Hyperbolic tangent \\
\midrule
  Output activation function & Sigmoid \\
\midrule
  Effect parameter low-pass filter cutoff frequency & 50 Hz \\
\midrule
  Fitness function & Local similarity \\
\end{longtable}
\end{center}

% !TEX encoding = UTF-8 Unicode
%!TEX root = thesis.tex
% !TEX spellcheck = en-US
%%=========================================
\section{Experiment 1}
In this experiment, the aim is to find a good combination of crossover rate and mutation rate.

\subsection{Configuration}
\begin{minipage}{\linewidth}
\centering
\captionof{table}{Table Title TODO} \label{tab:title} 
\begin{tabular}{ C{3.5in} C{1.6in} }\toprule[1.5pt]
\bf Parameter & \bf Value \\
\midrule
  Number of generations & 20 \\
\midrule
  Fitness function & Local similarity \\
\midrule
  Target sound & Drum loop \\
\midrule
  Input sound & White noise \\
\midrule
  Effect & Distortion and resonant low-pass filter \\
\midrule
  Audio features & mfcc\_0, mfcc\_0\_\_derivative, mfcc\_1 \\
\midrule
  Number of runs & 250 per configuration \\
\bottomrule[1.25pt]
\end {tabular}\par
\bigskip
Should be a caption TODO
\end{minipage}

\subsection{Fitness function}
The local similarity fitness function is based on the average euclidean distance between the feature vector of the target sound and the output sound in the k frames of the two sounds.

\begin{verbatim}
Function LOCAL_SIMILARITY(target, individual):
    total_euclidean_distance = 0
    for each k in range(num_frames):
        A = target.get_feature_vector(k)
        C = individual.get_feature_vector(k)
        total_euclidean_distance += EUCLIDEAN_DISTANCE(A, C)
    avg_euclidean_distance = total_euclidean_distance / num_frames
    return 1 / (1 + avg_euclidean distance)
\end{verbatim}

where \texttt{EUCLIDEAN\_DISTANCE} is $d(p,q)=\sqrt{(q_1-p_1)^2+(q_2-p_2)+...+(q_n-p_n)^2}$

\subsection{Evaluation of configurations}
Figure 4.1 TODO shows that ...

\begin{figure}[h]
    \centering
    \includegraphics[width=0.99\textwidth]{grid_search_crossover_mutation_avg}
    \caption{TODO caption}
    \label{fig:grid_search_crossover_mutation_avg}
\end{figure}

% !TEX encoding = UTF-8 Unicode
%!TEX root = thesis.tex
% !TEX spellcheck = en-US
%%=========================================
\section{Experiment 2}
In this experiment, the aim is to find a good value for add link probability et al TODO

\subsection{Configuration}



\begin{center}
\begin{longtable}{p{5cm} p{7cm}}
\caption[Experiment configuration]{Experiment configuration} \label{tab:exp1_configuration} \\

\hline \multicolumn{1}{l}{\textbf{Parameter}} & \multicolumn{1}{l}{\textbf{Value}} \\ \hline 
\endfirsthead

\multicolumn{2}{c}%
{{\bfseries \tablename\ \thetable{} -- continued from previous page}} \\
\hline \multicolumn{1}{l}{\textbf{Parameter}} & \multicolumn{1}{l}{\textbf{Value}} \\ \hline 
\endhead

\hline \multicolumn{2}{r}{{Continued on next page}} \\ \hline
\endfoot

\hline \hline
\endlastfoot

Number of generations & 50 \\
\midrule
Fitness function & Local similarity, as described in experiment 1 \\
\midrule
Target sound & Drum loop \\
\midrule
Input sound & White noise \\
\midrule
Effect & Distortion and resonant low-pass filter \\
\midrule
Audio features & mfcc\_0, mfcc\_0\_\_derivative, mfcc\_1 \\
\midrule
Number of runs & 400 per configuration \\
\end{longtable}
\end{center}

\subsection{Evaluation of configurations}
Figure \ref{fig:add_link_probability} TODO shows that 0.03 is probably the best value while 0.3 is significantly worse

\begin{figure}[h]
    \centering
    \includegraphics[width=0.99\textwidth]{add_link_probability}
    \caption{TODO caption}
    \label{fig:add_link_probability}
\end{figure}

TODO: Show typical end-result neural networks from all the configurations, to highlight that higher probability builds a larger, more complex network
% !TEX encoding = UTF-8 Unicode
%!TEX root = thesis.tex
% !TEX spellcheck = en-US
%%=========================================
\section{Experiment 3}
When using an evolved cross-adaptive audio effect in a live performance, a performer may want to use it in an expressive way. For example, if the performer is a drummer, he/she can vary the intensity of the drum hits. For the cross-adaptive audio effect to handle this, it needs to be trained on all the different intensities of the drum hits. If an extensive recording is available, that is fine. However, if the available target sound is short or lacks sufficient variation, one can harness the concept of data augmentation to create artificial variations of that sound. If one uses that sound instead, the evolved effect will typically be more capable of dealing with the generated variations. This experiment is about testing the author's implementation of data augmentation and the ability to apply evolved cross-adaptive audio effects to unseen sounds.

\subsection{Configuration}

\begin{center}
\begin{longtable}{p{5cm} p{9cm}}
\caption[Experiment configuration]{Experiment configuration} \label{tab:exp3_configuration} \\

\hline \multicolumn{1}{l}{\textbf{Parameter}} & \multicolumn{1}{l}{\textbf{Value}} \\ \hline 
\endfirsthead

\multicolumn{2}{c}%
{{\bfseries \tablename\ \thetable{} -- continued from previous page}} \\
\hline \multicolumn{1}{l}{\textbf{Parameter}} & \multicolumn{1}{l}{\textbf{Value}} \\ \hline 
\endhead

\hline \multicolumn{2}{r}{{Continued on next page}} \\ \hline
\endfoot

\hline \hline
\endlastfoot

Number of generations & 20 \\
\midrule
Target sound (training) & Drum loop with bass drum, snare drum, clap and hihat (figure \ref{fig:exp3_waveforms}) \\
\midrule
Target sound (validation) & Snare roll (rapid snare drum hits) with ascending pitch and amplitude (figure \ref{fig:exp3_waveforms}) \\
\midrule
Input sound & White noise \\
\midrule
Effect & Distortion and resonant low-pass filter \\
\midrule
Audio features & Root Mean Square (RMS) and spectral centroid \\
\midrule
Number of runs & 40 per configuration \\
\end{longtable}
\end{center}

\begin{figure}[H]
    \centering
    \includegraphics[width=1.0\textwidth]{exp3_waveforms}
    \caption{Waveform of training sound (top) and validation sound (bottom)}
    \label{fig:exp3_waveforms}
\end{figure}

The augmented variant of the training sound was created by repeating the sound 8 times, with variations in playback speed and gain for each repetition. The playback speed and gain are obtained by sampling from a gaussian distribution with standard deviations of $0.3$ and $0.5$, respectively.

Runs with three different target sounds (training sound, augmented training sound and validation sound) will be compared. The resulting effects are applied to the validation sound, and fitness values are measured. The neural networks trained on the validation sound are of course expected to be yield the highest fitness scores when tested on the validation sound. That configuration is included for comparison, as an upper bound estimate.

\subsection{Results and Evaluation}
Figure \ref{fig:exp3_fitness_box} shows that neural networks trained on an augmented variant of the training sound generalize better than neural networks trained on the nonaugmented training sound. In other words, the resulting audio effects becomes better at dealing with nuances of the situations in the training sound. There's a trade-off, however: Training on an augmented sound requires more computational time because there's more data to process. This may not be a problem, because training can be done before live performance starts.

\begin{figure}[H]
    \centering
    \includegraphics[width=1.0\textwidth]{exp3_fitness_box}
    \caption{Box plot of validation fitness values in final generation. The labels on the x-axis indicate which sound was used as target sound.}
    \label{fig:exp3_fitness_box}
\end{figure}

% !TEX encoding = UTF-8 Unicode
%!TEX root = thesis.tex
% !TEX spellcheck = en-US
%%=========================================
\section{Experiment 4}
In this experiment, the aim is to compare two different collections of audio features used in the similarity measure

Configuration A RMS, pitch and spectral centroid

Configuration B RMS, pitch, spectral centroid and bark bands

\subsection{Configuration}
\begin{minipage}{\linewidth}
\centering
\captionof{table}{Table Title TODO} \label{tab:exp4_configuration} 
\begin{tabular}{ C{2.5in} C{2.6in} }\toprule[1.5pt]
\bf Parameter & \bf Value \\
\midrule
  Number of generations & 500 \\
\midrule
  Fitness function & Local similarity \\
\midrule
  Target sound & Sine wave, 440 Hz \\
\midrule
  Input sound & White noise \\
\midrule
  Effect & Bandpass filter with up to 10x post gain \\
\midrule
  Audio features & \textbf{Configuration A}: RMS, pitch and spectral centroid \newline
  \textbf{Configuration B}: RMS, pitch, spectral centroid and 25 bark bands \\
\midrule
  Number of runs & 10 per configuration \\
\bottomrule[1.25pt]
\end {tabular}\par
\bigskip
Should be a caption TODO
\end{minipage}

\subsection{Evaluation}
Since fitness functions were different in these two configurations, the fitness values are not directly comparable. Instead, the results were evaluated by manually listening to the output sounds. In the first configuration, the results were fairly bad: All of the sounds were too noisy, and the author failed to perceive the tone. However, in terms of spectral centroid and amplitude, the sounds were a good match. In order to transform noise into a sine, the bandwidth of the bandpass filter has to be very narrow. A narrow filter would have lowered the overall amplitude of the sound. This would have been deemed bad by the fitness function, hence the genetic algorithm didn't effectively explore that area in the solution space. Also, a narrow filter wouldn't have yielded any improvements in the similarity in spectral centroid and/or pitch. Therefore, the typical solution has a broad bandpass filter, albeit with an appropriate center frequency. See in figure \ref{fig:exp4_spectrum_plot} that the peak frequency of the typical output sound matches well the peak frequency of the target sound.

The results in the second configuration, were much better. The author could hear a clear tone in all 10 output sounds. There was still some noise in most sounds. The author believes that the solutions would have improved with more generations, because the fitness was typically still increasing towards the 500th (last) generation. One of the output sounds featured vibrato (varying pitch over time), due to a noisy input being mapped to the center frequency parameter. This could probably have been alleviated by adding the derivative of the pitch as a dimension in the fitness function, so the unwanted vibrato would be punished more severely by the fitness function.

\begin{figure}[h]
    \centering
    \includegraphics[width=0.99\textwidth]{exp4_spectrum_plot}
    \caption{TODO caption TODO refer to this figure}
    \label{fig:exp4_spectrum_plot}
\end{figure}

The takeaway from this experiment is that
\begin{itemize}  
\item the collection of features used for similarity measures has a high impact on the resul
\item bark bands are useful
\end{itemize}
% !TEX encoding = UTF-8 Unicode
%!TEX root = thesis.tex
% !TEX spellcheck = en-US
%%=========================================
\section{Experiment 5}
This experiment is about combining several effects in serial and parallel. The hypothesis is that a genetic algorithm could be adept at choosing which effects to use and how.

For baseline performance measure, each effect has been tested separately. Then they were run in parallel TODO



In this experiment, we do 5 fx in parallel => ok result
am
bandpass
bitreduce
dist lpf
chorus

(Then 2 layers of 5?)
Then 10 fx in parallel => bad result

Suggest pre-training with each fx separately, then training the mix
Softmax mixing may also be bad -> suggest independent mix values

create video demonstrating 1st successful result here

Say something about hypothesis about which fx will be used. Which ones were actually used? Do the same for parameters. Show mix values in horizon graph. Say that am and chorus make sound richer, while we need it filtered.

\begin{figure}[h]
    \centering
    \includegraphics[width=0.45\textwidth]{exp5_1layer_softmax}
    \caption{Gain values for each effect in the best result with configuration 1 TODO}
    \label{fig:exp5_1layer_softmax}
\end{figure}

\begin{figure}[h]
    \centering
    \includegraphics[width=0.45\textwidth]{exp5_2layers_softmax}
    \caption{Gain values for each effect in the best result with configuration 2 TODO}
    \label{fig:exp5_2layers_softmax}
\end{figure}

\begin{figure}[h]
    \centering
    \includegraphics[width=0.99\textwidth]{exp5_avg_max}
    \caption{Aggregated fitness values}
    \label{fig:exp5_avg_max}
\end{figure}

\begin{figure}[h]
    \centering
    \includegraphics[width=0.99\textwidth]{exp5_box}
    \caption{Box-and-whiskers plot of fitness values in the last generation}
    \label{fig:exp5_box}
\end{figure}

